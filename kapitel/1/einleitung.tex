% \section{Einleitung}

% \subsection{Hintergrund und Motivation}

% Internet wurde 1991 erfunden (Quelle), und war ein kleines Netzwerk, welches von Forschern und dem Mitlitär genutzt wurde.

% Durch die Entwicklung von Betriebssystemen mit grafischen Oberflächen, 
% und der Einführung benutzerfreundlicher Browser (Quelle) wurde das Internet für die breite Masse zugänglich. 
% In den 90er Jahren (Quelle) gab es nur eine Handvoll Webseiten, 
% wie beispielsweise einge grundlegende Webseite mit Informationen zum WWW. 
% Im Dotcom Boom kamen Kommerziellen Internetanbietern auf was dazu fürte das viele Online Unternehmen wie 
% Paypal, Ebay, oder Yahoo anfingen das Internet für sich zu nutzen. 
% Auch die verbreitung von internetfähigen Mobilgeräten hatten einen erheblichen einfluss auf die Webseiten Entwicklung

% \begin{figure}[H]
% \caption{Rahmenbedingungen für die Literraurrecherche nach vom Brocke et al.}

% \end{figure}


% \subsection{Forschungsfrage}

% - Welche Vorteile und Nachteile haben Single Page Applikationen für den Besucher einer Webseite? 

% \subsection{Forschungsmethodik}

% \subsection{Abgrenzung}

% es wird nicht bahndelt wie: 
% eine Webseite gehostet wird
% wie die stärken und schwächen der technologien kombiniert werden können

% \subsection{Aufbau der Arbeit}


% Was sind Webseiten
% Welche Arten von Webseiten gibt es 
% Was versteht man unter Nutzererfahrung
% Welche Faktoren nehmen einfluss auf die Nutzererfahrung:::
% Was versteht man unter Ladezeit
% Was ist die initiale Ladezeit
% Was wird initial geladen


\section{Einleitung}

\subsection{Motivation und Problemstellung}
\subsection{Forschungsfrage und -methode}
Wie beeinflussen SPAs und traditionelle Webseiten die Nutzererfahrung in Bezug auf Ladezeiten?
\subsection{Aufbau der Arbeit}