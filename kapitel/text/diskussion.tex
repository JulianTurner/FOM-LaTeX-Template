\section{Diskussion}
In diesem Kapitel werden die Ergebnisse der Arbeit diskutiert, die Forschungsfrage beantwortet und die Implikationen der Ergebnisse erläutert.

\subsection{Zusammenfassung}
In dieser Arbeit wurden die Stärken und Schwächen von \ac{SPA} und traditionellen Webseiten untersucht.
Die Ergebnisse zeigen das eine \ac{SPA} ihre Stärken bei der Entwicklung von komplexen Webanwendungen ausspielen kann und
gerade bei häufigen Nutzerinteraktionen eine bessere Nutzererfahrung bietet in dem sie sowohl den \gls{overhead} in der Kommunikation mit dem \gls{server} reduziert sondern auch noch ohne \ac{UI} refresh auskommt.\footcite[Vgl.][Seite 5]{Solovei2018} \footcite[Vgl.][Seite 143]{Irudayaraj2019}
Eine \ac{SPA} kann auch ohne den \gls{server} funktionieren, nachdem alle Daten zur Verfügung stehen \footcite[Vgl.][Seite 6]{Solovei2018}
Doch eine \ac{SPA} ist komplexer zu entwickeln und benötigt eine komplexere Softwarearchitektur als eine traditionelle Webseite. \footcite[Vgl.][Seite 4]{Smith2022} \footcite[Vgl.][Seite 15]{Flanagan2011}


\subsection{Beantwortung der Forschungsfrage}
Die Forschungsfrage lautet: Welche Herausforderungen haben Entwickler von \ac{SPA} Websites im Vergleich zu traditionellen Mehrseiten-Websites?

Die Entwickler von \ac{SPA} Websites haben im Vergleich zu traditionellen Mehrseiten-Websites ganz andere Herausforderungen.
\subsubsection{Herausforderungen von \ac{SPA} Websites}
\paragraph*{Ladezeit}
Die initiale Ladezeit der \ac{SPA} gering zu halten um von Anfang an eine gute Performance der Webseite zu bietenV
\paragraph*{Framework}
Eine \ac{SPA} wird mit einem Framework umgesetzt, welches nicht ohne weiteres ausgetauscht werden kann. Sollte das
Framework nicht mehr weiterentwickelt werden, muss die \ac{SPA} auf ein anderes Framework migriert werden um es aktuell zu halten.
\paragraph*{Logik im Browser}
Die Logik der \ac{SPA} läuft im Browser und nicht auf dem Server. Dies kann zu Problemen führen, wenn der Browser des Nutzers
nicht alle Funktionen unterstützt. Hier muss entweder auf die Nutzergruppe verzichtet werden oder eine alternative Lösung für den Rückfall gefunden werden.
V\paragraph*{\ac{URL}}
Das Routing der \ac{URL} ist komplexer als bei traditionellen Webseiten. Hier muss sichergestellt werden, dass die \ac{URL}
auf den richtigen Zustand der \ac{SPA} zeigtV
\paragraph*{Komplexität}
Die Entwicklung einer \ac{SPA} ist komplexer als die Entwicklung einer traditionellen Webseite. Hier muss sichergestellt werden,
dass die Entwickler die Komplexität der \ac{SPA} beherrschen.

\subsection{Implikationen der Ergebnisse}
Viele Webseiten haben keine starke Nutzerinteraktionen sondern präsentieren eher statische Informationen und können daher
einfacher als eine traditionelle Webseite umgesetzt werden.
Ein Benefit der traditionellen Webseite ist, dass sie von Suchmaschinen besser indexiert werden können, da sie keine Logik benötigen um Daten anzuzeigen.
Da keine Logik im \gls{browser} ausgeführt werden muss, kann die Webseite gerade auf leistungsschwächeren Geräten schneller geladen werden.

Eine \ac{SPA} bietet eine bessere Nutzererfahrung bei komplexen Webanwendungen mit vielen Nutzerinteraktionen und benötigen kein \ac{UI} Refresh nach einer Nutzerinteraktion.
Des weiteren kann eine \ac{SPA} Daten direkt vom \gls{server} laden und muss nicht die komplette Webseite neu laden was sich für den Nutzer flüssiger nahtlos anfühlt.
Auch kann eine \ac{SPA} im Offline Modus funktionieren, wenn sie dafür ausgelegt wurde.

Eine traditionelle Webanwendung bieten sich an wenn man eine Webseite mit vielen Informationen und wenig Nutzerinteraktionen erstellen möchte.
Eine \ac{SPA} bietet sich an wenn man eine Webseite mit vielen Nutzerinteraktionen und einem flüssigen Nutzererlebnis erstellen möchte.
Sollte die Webseite ohne Logik in dem \gls{browser} laufen, fällt die Entscheidung auf eine traditionelle Webseite hingegen
soll die Webseite Inhalte ohne den \gls{server} darstellen, fällt die Entscheidung auf eine \ac{SPA}.