\section{Diskussion}
\subsection{Zusammenfassung}
Zusammenfassung der wichtigsten Erkenntnisse aus den Herausforderungen
\subsection{Beantwortung der Forschungsfrage}

\subsection{Implikationen der Ergebnisse}
Viele traditionelle Webseiten haben keine starke Nutzerinteraktionen sondern präsentieren eher Informationen und können daher
einfacher als eine \ac{SPA} umgesetzt werden. Ein Benefit der traditionellen Webseite ist, dass sie von Suchmaschinen besser indexiert werden können, da sie mehrere Seiten haben.
Da keine Logik im \gls{browser} ausgeführt werden muss, kann die Webseite gerade auf leistungsschwächeren Geräten schneller geladen werden.
\ac{SPA} bieten eine bessere Nutzererfahrung bei komplexen Webanwendungen mit vielen Nutzerinteraktionen und benötigen kein \ac{UI} Refresh bei Nutzerinteraktionen.
\ac{SPA} kann Daten direkt vom \gls{server} laden und muss nicht die komplette Webseite neu laden was sich für den Nutzer flüssiger anfühlt.
Auch kann eine \ac{SPA} im Offline Modus funktionieren, wenn sie dafür ausgelegt wurde.

Eine traditionelle Webanwendung bieten sich an wenn man eine Webseite mit vielen Informationen und wenig Nutzerinteraktionen erstellen möchte.
Eine \ac{SPA} bietet sich an wenn man eine Webseite mit vielen Nutzerinteraktionen und einem flüssigen Nutzererlebnis erstellen möchte.
Sollte die Webseite ohne Logik in dem \gls{browser} laufen, fällt die Entscheidung auf eine traditionelle Webseite hingegen
soll die Webseite Inhalte ohne den \gls{server} darstellen, fällt die Entscheidung auf eine \ac{SPA}.

\subsection{Reflexion der Arbeit}
Kritische Reflexion über die gewählte Methodik und potenzielle Limitationen der Literaturrecherche
Kaum relevante Literatur zu dem Thema gefunden
Oft nur sehr oberflächliche Informationen zu dem Thema
Oft nur ältere Literatur zu dem Thema
Meist nur Lehrbücher zu dem Thema