% Stellen Sie sicher, dass die Motivation und das Problem in einem klaren Zusammenhang zur Forschungsfrage stehen. Warum ist es wichtig, die Stärken und Schwächen von SPAs und traditionellen Webseiten zu untersuchen?

\section{Einleitung}
In der Einleitung wird das Thema der Arbeit eingeführt und die Motivation für die Arbeit dargelegt. 
Die Einleitung sollte den Leser dazu befähigen, die Arbeit in den Kontext der Forschungsliteratur einzuordnen.
Zudem sollte die Einleitung die Forschungsfrage und die Zielsetzung und einer Abgrenzung der Arbeit klar darlegen.
Darüber hinaus gibt die Einleitung den Aufbau der Arbeit wieder. 

\subsection{Hintergrund und Motivation}
% Einführung in das Thema der SPA-Websites und traditionellen Mehrseiten-Websites
% Bedeutung und Relevanz des Vergleichs der beiden Ansätze
Zu Beginn des Internet, als das Internet noch in den Kinderschuhen steckte, bestanden Webseiten meist aus statischen \ac{HTML} Seiten.
Die statischen Webseiten zeigten oft nur wie eine digitale Visitenkarte die Kontaktdaten, Produkte oder angebotene Services des Unternehmens an.
Diese digitalen Visitenkarten zielten nicht darauf ab einen Profit mit und durch die Webseite zu erzielen. \footcite[Vgl.][Seite 1]{Bly2018}
Mittlerweile ist das Internet zu einem wichtigen Bestandteil des Lebens geworden und wird von vielen Menschen täglich genutzt.
Durch die tägliche Nutzung wurde das Internet für viele zur primären Informationsquelle und einem wichtigen Kommunikationsmittel. \footcite[Vgl.][Seite 1]{conf/pi/Sassenberg}
Mit der transformation des Internet haben sich auch Webseiten verändert, welche sich mittlerweile in Anwendungen verwandelt haben und aktiv am Geschäft des Unternehmens beteiligt sind.
Die neuen Anwendungen sollten für die Betreiber einen Profit erzielen, in dem sie das Branding des Unternehmens darstellen, Informationsquelle für die Nutzer sind vor allem aber eine starke Bindung zwischen dem Unternehmen und den Nutzern herstellen und aufrechterhalten. \footcite[Vgl.][Seite 1]{Bly2018}
Der Bedarf nach Webseiten und die Erwartungen an die Webseite scheint groß \footcite[Vgl.][Seite 1]{Smith2022}, das belegt auch eine Umfrage der Community \Citeauthor{StackOverflow2023}, welche jährlich Softwareentwickler befragt welche Programmiersprachen und Technologien aktuell verwendet werden .
In der Umfrage von \Citeauthor{StackOverflow2023} gaben  53\% der Befragten an, dass sie mit \ac{HTML} und \ac{CSS} arbeiten, was für die Entwicklung für Webseiten aller Art relevant ist. \footcite[Vgl.][most-popular-technologies-programming-scripting-and-markup-languages]{StackOverflow2023}
Die im Vergleich zu den digitalen Visitenkarten neuartigen Anwendungen werden als traditionelle Webseiten, Webanwendungen, Mehrseiten-Webseiten oder \ac{SPA} Webseiten bezeichnet, welche sich in ihrer Funktionsweise stark von den statischen Webseiten unterscheiden.
Die traditionellen Webseiten bestehen aus mehreren \ac{HTML} Seiten, welche bei einem Klick auf einen Link oder Button eine komplett neue \ac{HTML} Seite laden.
Die \ac{SPA} hingegen besteht aus einer einzigen \ac{HTML} Seite, welche bei einem Klick auf einen Link oder Button nur die Inhalte der Seite im \gls{browser} verändert und nicht zu einer neuen Seite navigiert.
Durch diese Veränderung erhofft man sich Gewinne im Bereich der Performance, da die Webseite nicht bei jedem Klick eine neue Seite laden muss, sondern nur die Inhalte der Seite verändert.
So wie mit jeder Technologie haben beide Ansätze verschiedene Umsetzungen und Implementierungen, was zu verschiedenen Meinungen und Ansichten von \ac{SPA} führt. \footcite[Vgl.][Seite 4]{Flanagan2011}
Das Ziel dieser Arbeit ist es, die Vor- und Nachteile der traditionellen Webseiten und der \ac{SPA} Webseiten herauszuarbeiten und zu vergleichen,
um Entwicklern eine Entscheidungsgrundlage für die Entwicklung von Webseiten zu geben.

\subsection{Problemstellung und Zielsetzung der Literaturrecherche}
Das Entwicklern einer Webseite oder einer Webanwendungen kann eine sehr komplexe Aufgabe sein, da es viele verschiedene Ansätze und Technologien gibt, welche sich selbst auch ständig weiterentwickeln.
Durch die ständige Weiterentwicklung der Technologien und Ansätze ist es für Entwickler nicht immer einfach den Überblick zu behalten und die richtige Entscheidung zu treffen.
Entwickler müssen sich mit den verschiedenen Ansätzen und Technologien auseinandersetzen und diese vergleichen, um die richtige Entscheidung für die Entwicklung der Webseite zu treffen.
Dabei gilt es zu Verstehen was das Ziel der Webseite ist und welche Anforderungen die Webseite erfüllen muss um eine optimale Nutzererfahrung zu bieten.
Die Thema dieser Arbeit ist es, die Vor- und Nachteile der traditionellen Webseiten und der \ac{SPA} Webseiten herauszuarbeiten und zu vergleichen.
Die Problemstellung ist aus relevant für Entwickler, da die Vor- und Nachteile der beiden Ansätze einen Einfluss auf die Nutzererfahrung der Webseite haben.
Die Forschungsfrage dieser Arbeit lautet: Welche Herausforderungen haben Entwickler bei \ac{SPA} Websites im Vergleich zu traditionellen Mehrseiten-Websites?

\subsection{Abgrenzung}
Nicht Teil dieser Arbeit ist es eine klare Entscheidungsgrundlage einer \ac{SPA} Website und einer traditionellen Mehrseiten-Website zu untersuchen.
Diese Abgrenzung wurde getroffen da es nicht möglich ist eine klare Entscheidungsgrundlage zu finden, da die Wahl von vielen Faktoren wie die Größe der Webseite,
Anzahl der Besucher, Entwicklungskosten und vielen weiteren abhängt.

\subsection{Aufbau der Arbeit}

Die Arbeit wurde in sechs Kapitel unterteilt, um den roten Faden zu wahren.
Im ersten Kapitel befindet sich die Einleitung, in welcher das Thema der Arbeit eingeführt und die Motivation, Problemstellung sowie die Abgrenzung für die Arbeit dargelegt wird.
Das Grundlagenkapitel wird verwendet, um Grundlagen und Konzepte zu erläutern, welche für das Verständnis der Arbeit notwendig sind.
Das dritte Kapitel ist das Kapitel der Methodik, in der das Vorgehen nach \Citeauthor{conf/ecis/BrockeSNRPC09} beschreiben ist.
Im vierten Kapitel werden die Stärken und Schwächen von \ac{SPA} und traditionellen Webseiten aufgeführt, welche im Rahmen der Literaturrecherche gefunden wurden.
In der Diskussion, werden die Ergebnisse der Literaturrecherche diskutiert, die Forschungsfrage beantwortet, Implikationen der Ergebnisse für die Praxis aufgezeigt und abschließend für das Kapitel die Arbeit reflektiert.
Die Arbeit endet mit dem Fazit, wo die Ergebnisse der Arbeit zusammengefasst werden und ein Ausblick auf die Zukunft gegeben wird.
