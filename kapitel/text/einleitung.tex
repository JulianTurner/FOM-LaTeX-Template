% Stellen Sie sicher, dass die Motivation und das Problem in einem klaren Zusammenhang zur Forschungsfrage stehen. Warum ist es wichtig, die Stärken und Schwächen von SPAs und traditionellen Webseiten zu untersuchen?

\section{Einleitung}
In der Einleitung wird das Thema der Arbeit eingeführt und die Motivation für die Arbeit dargelegt. 
Die Einleitung sollte den Leser dazu befähigen, die Arbeit in den Kontext der Forschungsliteratur einzuordnen.
Zudem sollte die Einleitung die Forschungsfrage und die Zielsetzung der Arbeit klar darlegen.
Darüber hinaus gibt die Einleitung den Aufbau der Arbeit wieder. 

\subsection{Hintergrund und Motivation}
% Einführung in das Thema der SPA-Websites und traditionellen Mehrseiten-Websites
% Bedeutung und Relevanz des Vergleichs der beiden Ansätze
Zu Beginn des Internet, als das Internet noch in den Kinderschuhen steckte, bestanden Webseiten aus statischen \ac{HTML} Seiten.
Die statischen Webseiten zeigten meist nur eine Visitenkarte mit den Kontaktdaten, Produkten oder Services des Unternehmens an,
zielten aber nicht darauf ab, ein Profit mit und durch die Webseite zu erzielen. \footcite[Vgl.][Seite 1]{Bly2018}
Mittlerweile ist das Internet ist ein wichtiger Bestandteil des täglichen Lebens geworden und wird von vielen Menschen täglich genutzt,
und wurde zur primären Informationsquelle sowie zu einem sehr wichtigen Kommunikationsmittel, \footcite[Vgl.][Seite 1]{conf/pi/Sassenberg}
Webseiten verwandeln sich in Anwendungen welche aktiv am Geschäft des Unternehmens beteiligt sind und einen Profit erzielen sollen indem die Webseiten
das Branding des Unternehmens darstellen, Informationsquelle für die Nutzer sind vor allem aber eine starke Verbindung zwischen dem Unternehmen und den Nutzern herstellen. \footcite[Vgl.][Seite 1]{Bly2018}

\subsection{Zielsetzung der Literaturrecherche}
Welche Herausforderungen haben Entwickler bei \ac{SPA} Websites im Vergleich zu traditionellen Mehrseiten-Websites?

\subsection{Aufbau der Arbeit}
