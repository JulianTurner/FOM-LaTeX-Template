\section{Methodik}
Die Wissenschaft ist ein kumulatives Unterfangen, da neues Wissen oft erst durch die Interpretation und Kombination von vorhandenem Wissen entsteht.
Aus diesem Grund spielen Literaturrecherchen seit langem eine entscheidende Rolle in der Wissenschaft.\footcite[Vgl.][Seite 1]{conf/ecis/BrockeSNRPC09}
Die Qualität von einer Literaturrecherche wird nach \citeauthor{conf/ecis/BrockeSNRPC09} insbesondere durch den Prozess der Literatursuche bestimmt.
In dieser Arbeit sollen Herausforderungen und Vorteile für Entwickler bei der Implementierung von \ac{SPA} Websites im Vergleich zu traditionellen Mehrseiten-Websites auf Grundlage von bereits bestehender Literatur und Forschung erfolgen.
Um die einzelnen Herausforderungen und Vorteile für Entwickler zu identifizieren, wird eine systematische Literaturrecherche durchgeführt.
Neben des erstellen des Artefakts dieser Arbeit ist es auch notwendig die Literaturrecherche und die Literaturanalyse zu dokumentieren, um die Nachvollziehbarkeit der Arbeit zu gewährleisten.
Neben der Nachvollziehbarkeit ist es auch wichtig, dass sowohl die Literaturrecherche als auch Literaturanalyse reproduzierbar sind.
Der Fokus liegt hierbei vor allem auf Veröffentlichung welche sich mit den Themen \ac{SPA} und traditionellen Webseiten beschäftigen oder einen Vergleich der beiden Themen durchführen.
Die systematische Literaturrecherche orientiert sich an den Richtlinien von \citeauthor{conf/ecis/BrockeSNRPC09} und lässt sich dabei in fünf Phasen unterteilen.
In der ersten Phase wird die Forschungsfrage definiert, welche in dieser Arbeit bereits in der Einleitung vorgestellt wurde, und legt dadurch dem Themenbereich fest.
In der zweiten Phase werden die Suchbegriffe sowie Ein- und Ausschlusskriterien definiert, welche in der dritten Phase verwendet werden, um die Literatur zu finden.
Die definition von Suchbegriffen sowie Ein- und Ausschlusskriterien findet in einem iterativen Prozess statt.
Die eigentliche Suche nach Literatur erfolgt in der dritten Phase, wobei die Suchbegriffe in verschiedenen Datenbanken verwendet werden.
In der vierten Phase werden die gefundenen Literaturquellen anhand konkreter Kriterien ausgewählt.
In der fünften und letzten Phase werden die ausgewählten Literaturquellen analysiert und die Ergebnisse dokumentiert. \footcite[Vgl.][Seite 2]{conf/ecis/BrockeSNRPC09}
Die Ergebnisse der Literaturrecherche werden in der Literaturanalyse zusammengefasst und die Herausforderungen und Vorteile der Entwickler bei der Implementierung von \ac{SPA} Websites im Vergleich zu traditionellen Mehrseiten-Websites identifiziert.
Die Ergebnisse werden neutral und sachlich betrachtet und ein Fachpublikum angesprochen. Damit eine möglichst objektive Betrachtung der Ergebnisse erfolgt, wurde jede Form von Literatur verwendet, welche sich mit dem Thema beschäftigt.

\subsection{Suchbegriffe}
Die Suchbegriffe wurden in einem iterativen Prozess definiert und angepasst.
Die Suchbegriffe wurden in drei Kategorien unterteilt, welche in der folgenden Tabelle dargestellt sind.
Die erste Kategorie beinhaltet Suchbegriffe, welche sich mit dem Thema \ac{SPA} beschäftigen.
Die zweite Kategorie beinhaltet Suchbegriffe, welche sich mit dem Thema traditionelle Webapplikation beschäftigen.
Die dritte Kategorie beinhaltet Suchbegriffe, welche sich mit dem Thema Vergleich von \ac{SPA} und traditionellen Webapplikation beschäftigen.
Die Suchbegriffe wurden in englischer Sprache definiert, da davon ausgegangen wird dass die meisten wissenschaftlichen Veröffentlichungen in englischer Sprache verfasst sind.

\begin{table}[H]
    \caption{Suchbegriffe}
    \label{tbl:suchbegriffe}
    \begin{tabularx}{\textwidth}[ht]{|X|}
        \hline
        \textbf{\ac{SPA}}                                                     \\
        \hline\hline
        \ac{SPA}                                                              \\
        Singe Page Application                                                \\
        modern Webapp OR Webseite OR Webpage                                  \\
        \hline\hline
        \hline
        \textbf{traditionelle Webseiten}                                      \\
        \hline\hline
        traditional Webapp OR Website OR Webpage                              \\
        traditional Webapps                                                   \\
        Multi Page application                                                \\
        Dynamic Webapplikation                                                \\
        \hline\hline
        \hline
        \textbf{Vergleich \ac{SPA} und traditionelle Webseiten}               \\
        \hline\hline
        Webapplikations                                                       \\
        Singe Page Application AND traditional Webpages                       \\
        SPA AND traditional Webpages AND challenges                           \\
        SPA AND Web AND drawbacks OR problems OR disadvantages OR limitations \\
        Websites AND performance OR speed OR loading time OR latency          \\
        \hline
    \end{tabularx}
\end{table}

\subsection{Durchführung der Literaturrecherche}
Um die relevant Literatur zu finden, wurden die Suchbegriffe in verschiedenen Datenbanken verwendet.
Es wurde im Rahmen des EBSCO Discovery Service nach Literatur gesucht.
Im EBSCO Discovery Service waren zum Zeitpunkt der Literaturrecherche folgende Datenbanken angebunden:
\begin{itemize}
    \item Business Source Ultimate
    \item EconLit
    \item APA PsycArticles
    \item PSYNDEX
    \item Medline
    \item CINAHL
    \item Engineering Source
    \item GreenFILE
    \item sowie Inhalte der \begin{itemize}
              \item IEEE Xplore Digital Library
              \item ACM Digital Library
              \item SpringerLink
              \item Thieme Connect
          \end{itemize}
\end{itemize}

Zusätzlich wurde im Rahmen der Literaturrecherche die Suchmaschinen Google Scholar und Research Gate verwendet.

\subsection{Auswahl der Literaturquellen}
Die Auswahl der Literaturquellen erfolgte in zwei Schritten.
Der erste Schritt bestand darin, die gefundenen Literaturquellen anhand konkreter Kriterien auszuwählen.
Der zweite Schritt bestand darin, die ausgewählten Literaturquellen zu analysieren und die Ergebnisse zu dokumentieren.
Die Kriterien für die Auswahl der Literaturquellen wurden in einem iterativen Prozess definiert und angepasst.
Die folgenden Kriterien wurden für die Auswahl der Literaturquellen definiert:
\itemize{
    \item Das Werk:
    \begin{itemize}
        \item ist in englischer Sprache verfasst
        \item ist öffentlich verfügbar
        \item stammt aus nachvollziehbarer Quelle
        \item befasst sich mit \ac{SPA}
        \item befasst sich mit traditionellen Webseiten
        \item befasst sich mit dem Vergleich von \ac{SPA} und traditionellen Webseiten
    \end{itemize}
}
Die Literatur muss mindestens öffentlich verfügbar, in englischer Sprache verfasst und aus einer nachvollziehbaren Quelle stammen um berücksichtigt zu werden.


\subsection{Dokumentation der Ergebnisse}
Die Ergebnisse der Literaturrecherche werden in der Literaturanalyse zusammengefasst und die Herausforderungen der Entwickler von \ac{SPA} Websites im Vergleich zu traditionellen Mehrseiten-Websites identifiziert.
Die Ergebnisse werden neutral und sachlich betrachtet und ein Fachpublikum angesprochen.
Damit eine möglichst objektive Betrachtung der Ergebnisse erfolgt, wurde jede Form von Literatur verwendet, welche sich mit dem Thema beschäftigt.

\subsection*{Zusammenfassung der Methodik}
Die Literaturrecherche wurde nach den Richtlinien von \citeauthor{conf/ecis/BrockeSNRPC09} durchgeführt, um die Nachvollziehbarkeit und Reproduzierbarkeit der Arbeit zu gewährleisten.
Dazu wurde die Literaturrecherche in fünf Phasen unterteilt.
In der ersten Phase wurde die Forschungsfrage definiert, welche in dieser Arbeit bereits in der Einleitung vorgestellt wurde, und legt dadurch den Themenbereich fest.
In der zweiten Phase wurden die Suchbegriffe sowie Ein- und Ausschlusskriterien definiert, welche in der dritten Phase verwendet wurden, um die Literatur zu finden.
Die eigentliche Suche nach Literatur erfolgte in der dritten Phase, wobei die Suchbegriffe in verschiedenen Datenbanken verwendet wurden.
In der vierten Phase wurden die gefundenen Literaturquellen anhand konkreter Kriterien ausgewählt.
In der fünften und letzten Phase wurden die ausgewählten Literaturquellen analysiert und die Ergebnisse dokumentiert.