\section{Methodik}
Die Wissenschaft ist ein kumulatives Unterfangen, da neues Wissen oft erst durch die Interpretation und Kombination von vorhandenem Wissen entsteht. 
Aus diesem Grund spielen Literaturrecherchen seit langem eine entscheidende Rolle in der Wissenschaft. 
Die Qualität von einer Literaturrecherche wird insbesondere durch den Prozess der Literatursuche bestimmt.
\footcite{conf/ecis/BrockeSNRPC09}
In dieser Arbeit sollen Herausforderungen von \ac{SPA} Websites im Vergleich zu traditionellen Mehrseiten-Websites auf Grundlage von bereits bestehender Literatur, Forschung und Fallstudien erfolgen.
Um die einzelnen Herausforderungen zu identifizieren, wird eine systematische Literaturrecherche durchgeführt.
Neben des erstellen des Artefakts dieser Arbeit ist es auch notwendig die Literaturrecherche und die Literaturanalyse zu dokumentieren, um die Nachvollziehbarkeit der Arbeit zu gewährleisten.
Neben der Nachvollziehbarkeit ist es auch wichtig, dass die Literaturrecherche und die Literaturanalyse reproduzierbar sind. 
Dieses Kaptiel wird daher die Methodik dieser Literaturrecherche vorgestellt, welche sich an dem Vorgehen von \citeauthor{conf/ecis/BrockeSNRPC09} orientiert.
\subsection{Taxonomie der Arbeit}
\subsection{Literaturrecherche}
\subsection{Literaturanalyse}