\section{Methodik}
Die Wissenschaft ist ein kumulatives Unterfangen, da neues Wissen oft erst durch die Interpretation und Kombination von vorhandenem Wissen entsteht.
Aus diesem Grund spielen Literaturrecherchen seit langem eine entscheidende Rolle in der Wissenschaft.\footcite[Vgl.][Seite 1]{conf/ecis/BrockeSNRPC09}
Die Qualität von einer Literaturrecherche wird nach \citeauthor{conf/ecis/BrockeSNRPC09} insbesondere durch den Prozess der Literatursuche bestimmt.
In dieser Arbeit sollen Herausforderungen für Entwickler von \ac{SPA} Websites im Vergleich zu traditionellen Mehrseiten-Websites auf Grundlage von bereits bestehender Literatur, Forschung und Fallstudien erfolgen.
Um die einzelnen Herausforderungen für Entwickler zu identifizieren, wird eine systematische Literaturrecherche durchgeführt.
Neben des erstellen des Artefakts dieser Arbeit ist es auch notwendig die Literaturrecherche und die Literaturanalyse zu dokumentieren, um die Nachvollziehbarkeit der Arbeit zu gewährleisten.
Neben der Nachvollziehbarkeit ist es auch wichtig, dass sowohl die Literaturrecherche als auch Literaturanalyse reproduzierbar sind.
Der Fokus liegt hierbei vor allem auf Veröffentlichung welche sich mit den Themen \ac{SPA} und traditionellen Webseiten beschäftigen oder einen Vergleich der beiden Themen durchführen.
Die systematische Literaturrecherche orientiert sich an den Richtlinien von \citeauthor{conf/ecis/BrockeSNRPC09} und lässt sich dabei in fünf Phasen unterteilen.
In der ersten Phase wird die Forschungsfrage definiert, welche in dieser Arbeit bereits in der Einleitung vorgestellt wurde, und legt dadurch dem Themenbereich fest.
In der zweiten Phase werden die Suchbegriffe sowie Ein- und Ausschlusskriterien definiert, welche in der dritten Phase verwendet werden, um die Literatur zu finden.
Die definition von Suchbegriffen sowie Ein- und Ausschlusskriterien findet in einem iterativen Prozess statt.
Die eigentliche Suche nach Literatur erfolgt in der dritten Phase, wobei die Suchbegriffe in verschiedenen Datenbanken verwendet werden.
In der vierten Phase werden die gefundenen Literaturquellen anhand konkreter Kriterien ausgewählt.
In der fünften und letzten Phase werden die ausgewählten Literaturquellen analysiert und die Ergebnisse dokumentiert. \footcite[Vgl.][Seite 2]{conf/ecis/BrockeSNRPC09}
Die Ergebnisse der Literaturrecherche werden in der Literaturanalyse zusammengefasst und die Herausforderungen  der Entwickler von \ac{SPA} Websites im Vergleich zu traditionellen Mehrseiten-Websites identifiziert.
