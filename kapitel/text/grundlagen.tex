\section{Grundlagen}
\subsection{Definitionen}
\paragraph*{Websites}
Eine Webseite ist ein \ac{HTML} Dokument welche im \gls{browser} dargestellt werden kann. \footcite[Vgl.][Seite 27]{Robbins2018}
Ein \ac{HTML} Dokument besteht aus \ac{HTML} Elementen welche den Inhalt der Webseite beschreiben und \ac{CSS} Elementen welche das Aussehen der Webseite beschreiben. \footcite[Vgl.][Seite 29]{Robbins2018}
Das rendern der Webseite im \gls{browser} passiert auf einmal. Bei langsamen Verbindungen oder wenn die Webseite viele Medien Inhalte wie Bilder oder Videos beinhaltet kann es zu einer Verzögerung kommen bis die Webseite vollständig dargestellt wird. \footcite[Vgl.][Seite 31]{Robbins2018}
Grundsätzlich kann zwischen Statischen und Dynamischen Webseiten unterschieden werden. 
Statische Webseiten zeigen immer den gleichen Inhalt an, während Dynamische Webseiten den Inhalt je nach Anfrage an den \gls{server} ändern.

\paragraph*{Performance}
Unter Performance einer Webseite versteht man, wie schnell eine Webseite heruntergeladen und im \gls{browser} dargestellt wird. Die Performance wird durch die Größe der benötigten Dateien und der Anzahl der Anfragen an den \gls{server} 
beeinflusst und hat einen kritischen Einfluss wie Nutzer Zugang zu einer Webseite erhalten und mit der Webseite interagieren. \footcite[Vgl.][Seite 44]{Robbins2018}
Ein Experiment von \Citeauthor{Google2009} im Jahr 2009 hat ergeben dass eine Erhöhung der Latenzzeit bei der Websuche zwischen 100 und 400 Millisekunden die tägliche
Anzahl der Suchanfragen pro Nutzer um 0,2 \% bis 0,6\% reduziert. 
Neben den Ausbleiben der Suchanfragen führten die Nutzer weniger Suchvorgänge durch, je höher Latenz wurde. 
Selbst nachdem die Latenz wieder auf das vorherige Niveau gesunken ist, blieben die Suchanfragen noch für einige Zeit aus. \footcite[Vgl.][Seite 1]{Google2009}
\paragraph*{Ladezeiten}

\paragraph*{SEO}

\subsection{traditionelle Websites}
Eine traditionelle Webseite ist eine dynamische Webseite welche aus mehren \ac{HTML} Dokumenten besteht und auf dem \gls{server} ausgeführt und erzeugt wird, Daten aus einer Datenbank bezieht und für jeden Nutzer personalisiert werden kann. \footcite[Vgl.][Seite 32]{Robbins2018}
Eine typische Interaktion mit einer traditionellen Webseite starten mit dem Anfragen der Webseite vom \gls{browser} an den \gls{server}.
Der \gls{server} verarbeitet die Anfrage mit einem \gls{controller} und sendet eine Antwort mit dem ausgefüllten \ac{HTML} \gls{template} an den \gls{browser} als Antwort zurück.
Der \gls{browser} rendert die Webseite und zeigt sie dem Nutzer an. \footcite[Vgl. ][Seite 32]{Robbins2018}
Bei einer erneuten Interaktion mit der Webseite resultiert in eine neue Anfrage an den \gls{server} welcher die Anfrage wieder mit einem \gls{controller} verarbeitet und darauf mit
den neuen Daten das \ac{HTML} \gls{template} befüllt und dem \gls{browser} als Antwort zurück sendet und daraufhin die Webseite neu lädt. \footcite[Vgl. ][Seite 5]{Scott2015}


\subsection{\ac{SPA}}
Eine \ac{SPA} ist eine Anwendung, welche in einer einzigen Webseite läuft, wodurch der Code welcher zum befüllten der \ac{HTML} \gls{template} mit Daten verantwortlich ist, vom Server auf den \gls{browser} verschoben wurde. \footcite[Vgl.][Seite 4]{Scott2015}
Durch diese Verschiebung werden die Anfragen an den \gls{server} werden die Daten welche der \gls{browser} mit dem \gls{server} austauscht auf die Nutzdaten reduziert und das neu Laden der Webseite im \gls{browser} überflüssig macht. \footcite[Vgl.][Seite 7]{Scott2015}
