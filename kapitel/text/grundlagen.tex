\section{Grundlagen}
\subsection{Definitionen}
\paragraph*{Websites}
\paragraph*{Performance}
Die Performance von Webseiten wird durch die Größe der benötigten Dateien und der Anzahl der Anfragen an den \gls{server} 
beeinflusst und hat einen kritischen Einfluss wie Nutzer Zugang zu einer Webseite erhalten und mit der Webseite interagieren. \footcite[Vgl.][S. 44]{Robbins2018}
Ein Experiment von \Citeauthor{Google2009} im Jahr 2009 hat ergeben dass eine Erhöhung der Latenzzeit bei der Websuche zwischen 100 und 400 Millisekunden die tägliche
Anzahl der Suchanfragen pro Nutzer um 0,2 \% bis 0,6\% reduziert. 
Neben den Ausbleiben der Suchanfragen führten die Nutzer weniger Suchvorgänge durch, je höher Latenz wurde. 
Selbst nachdem die Latenz wieder auf das vorherige Niveau gesunken ist, blieben die Suchanfragen noch für einige Zeit aus. \footcite[Vgl.][S. 1]{Google2009}
\paragraph*{Ladezeiten}
\paragraph*{SEO}
\subsection{traditionelle Websites}
Definitionen und Abgrenzungen von SPAs und traditionellen Websites
\footcite[Vgl. ][Seite 1]{Scott2015}
\subsection{\ac{SPA}}
Definitionen und Abgrenzungen von SPAs und traditionellen Websites