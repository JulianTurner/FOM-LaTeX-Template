% \subsection{Diskussion}
% In diesem Kapitel werden die Ergebnisse der Arbeit diskutiert, die Forschungsfrage beantwortet und die Implikationen der Ergebnisse erläutert.

% \subsubsection*{Zusammenfassung}
% % In dieser Arbeit wurden die Stärken und Schwächen von \ac{SPA} und traditionellen Webseiten untersucht.
% % Die Ergebnisse zeigen das eine \ac{SPA} ihre Stärken bei der Entwicklung von komplexen Webanwendungen ausspielen kann und
% % gerade bei häufigen Nutzerinteraktionen eine bessere Nutzererfahrung bietet in dem sie sowohl den \gls{overhead} in der Kommunikation mit dem \gls{server} reduziert sondern auch noch ohne \ac{UI} refresh auskommt.
% % Eine \ac{SPA} kann auch ohne den \gls{server} funktionieren, nachdem alle Daten zur Verfügung stehen
% % Doch eine \ac{SPA} ist komplexer zu entwickeln und benötigt eine komplexere Softwarearchitektur als eine traditionelle Webseite.
% Diese Arbeit hat die Vor- und Nachteile von \ac{SPA} im Vergleich zu traditionellen Webseiten beleuchtet.
% Es wurde festgestellt, dass \ac{SPA} insbesondere bei komplexen Webanwendungen Vorteile bietet, indem sie eine reibungslose Nutzererfahrung ermöglichen, den Kommunikations-\gls{overhead} mit dem \gls{server} minimieren und ohne vollständige \ac{UI}-Aktualisierung auskommen.\cite[Vgl.][Seite 5]{Solovei2018} \cite[Vgl.]{Irudayaraj2019}
% Allerdings erfordert die Entwicklung einer \ac{SPA} eine tiefere Kenntnis spezifischer Frameworks und Architekturen, was sie komplexer macht als die Erstellung traditioneller Webseiten.\cite[Vgl.][Seite 4]{Smith2022} \cite[Vgl.][Seite 15]{Flanagan2011}


\subsection{Beantwortung der Forschungsfrage}
Eine Implementierung einer \ac{SPA} bringt sowohl Herausforderungen als auch Vorteile mit sich.
Durch die Analyse dieser Arbeit konnten folgende Herausforderungen und Vorteile identifiziert werden.

\paragraph*{Vorteile}
\begin{enumerate}
    \item schnelle und reibungslose Nutzererfahrung, da die Webseite nicht bei jeder Nutzerinteraktion eine neue Seite laden muss und nur Teile der Webseite aktualisiert
    \item \gls{overhead} wird aus der Kommunikation mit dem \gls{server} reduziert, da nur Nutzdaten zwischen \gls{server} und \gls{browser} ausgetauscht werden
    \item kann auch ohne Verbindung zum \gls{server} funktionieren, der \gls{server} wird nicht zur Präsentation der Daten benötigt wird
\end{enumerate}

\paragraph*{Herausforderungen}
\begin{enumerate}
    \item Ladezeit: Es ist entscheidend, die initiale Ladezeit einer \ac{SPA} zu minimieren, um von Beginn an eine hohe Performance zu gewährleisten.
    \item \gls{framework}-Abhängigkeit: \ac{SPA}s sind oft eng mit einem bestimmten \gls{framework} verbunden. Ein Wechsel oder eine Migration kann komplex und zeitaufwendig sein.
    \item Browser-Logik: Da die Logik einer \ac{SPA} im Browser ausgeführt wird, können Kompatibilitätsprobleme mit bestimmten Browsern auftreten.
    \item \ac{URL} Routing: Das Routing in einer \ac{SPA} kann komplexer sein als bei traditionellen Webseiten, insbesondere wenn es darum geht, den aktuellen Zustand der Anwendung korrekt widerzuspiegeln.
    \item Entwicklungskomplexität: Die Entwicklung einer \ac{SPA} erfordert ein tieferes Verständnis und Fachwissen als traditionelle Webseiten.
\end{enumerate}


% Die Entwickler von \ac{SPA} Websites haben im Vergleich zu traditionellen Mehrseiten-Websites ganz andere Herausforderungen.
% \subsubsection{Herausforderungen von \ac{SPA} Websites}
% \paragraph*{Ladezeit}
% Die initiale Ladezeit der \ac{SPA} gering zu halten um von Anfang an eine gute Performance der Webseite zu bietenV
% \paragraph*{Framework}
% Eine \ac{SPA} wird mit einem Framework umgesetzt, welches nicht ohne weiteres ausgetauscht werden kann. Sollte das
% Framework nicht mehr weiterentwickelt werden, muss die \ac{SPA} auf ein anderes Framework migriert werden um es aktuell zu halten.
% \paragraph*{Logik im Browser}
% Die Logik der \ac{SPA} läuft im Browser und nicht auf dem Server. Dies kann zu Problemen führen, wenn der Browser des Nutzers
% nicht alle Funktionen unterstützt. Hier muss entweder auf die Nutzergruppe verzichtet werden oder eine alternative Lösung für den Rückfall gefunden werden.
% V\paragraph*{\ac{URL}}
% Das Routing der \ac{URL} ist komplexer als bei traditionellen Webseiten. Hier muss sichergestellt werden, dass die \ac{URL}
% auf den richtigen Zustand der \ac{SPA} zeigtV
% \paragraph*{Komplexität}
% Die Entwicklung einer \ac{SPA} ist komplexer als die Entwicklung einer traditionellen Webseite. Hier muss sichergestellt werden,
% dass die Entwickler die Komplexität der \ac{SPA} beherrschen.

\subsection{Implikationen der Ergebnisse}
Traditionelle Webseiten eignen sich besonders für Inhalte mit geringer Interaktivität, da sie einfacher zu entwickeln sind und von Suchmaschinen besser indexiert werden können.
Ihre Performance kann auf leistungsschwächeren Geräten überlegen sein, da keine Logik im \gls{browser} erforderlich ist.

Eine \ac{SPA} hingegen bietet bei Webanwendungen mit intensiver Nutzerinteraktion ein überlegenes Nutzererlebnis.
Sie können Daten dynamisch und nahtlos vom \gls{server} abrufen und anzeigen, ohne die gesamte Seite neu laden zu müssen.
Wenn eine \ac{SPA} entsprechend entwickelt wurde, kann sie nach dem initialen Laden auch ohne Verbindung zum \gls{server} funktionieren.

Die Wahl zwischen einer \ac{SPA} und einer traditionellen Webseite sollte daher auf der Art der Anwendung, den spezifischen Anforderungen und dem beabsichtigten Publikum basieren.
Wenn eine Webseite primär Informationszwecken dient und wenig Interaktivität erfordert, könnte eine traditionelle Webseite die bessere Wahl sein.
Für Anwendungen, die eine hohe Interaktivität und ein nahtloses Nutzererlebnis erfordern, wäre eine \ac{SPA} vorzuziehen.
% Webseiten die keine starke Nutzerinteraktionen haben sondern eher statische Informationen präsentierten können einfacher als eine traditionelle Webseite umgesetzt werden.
% Ein Benefit der traditionellen Webseite ist, dass sie von Suchmaschinen besser indexiert werden können, da sie keine Logik benötigen um Daten anzuzeigen.
% Da keine Logik im \gls{browser} ausgeführt werden muss, kann die Webseite gerade auf leistungsschwächeren Geräten schneller zur Anzeige gebracht werden.

% Eine \ac{SPA} bietet eine bessere Nutzererfahrung bei komplexen Webanwendungen mit vielen Nutzerinteraktionen und benötigen kein \ac{UI} Refresh nach einer Nutzerinteraktion.
% Des weiteren kann eine \ac{SPA} Daten direkt vom \gls{server} laden und muss nicht die komplette Webseite neu laden was sich für den Nutzer flüssiger nahtlos anfühlt.
% Zusätzlich kann eine \ac{SPA} im Offline Modus funktionieren, wenn sie dafür ausgelegt wurde und alle notwendigen Daten hat.

% Eine traditionelle Webanwendung bieten sich an wenn man eine Webseite mit vielen Informationen und wenig Nutzerinteraktionen erstellen möchte.
% Eine \ac{SPA} bietet sich an wenn man eine Webseite mit vielen Nutzerinteraktionen und einem flüssigen Nutzererlebnis erstellen möchte.
% Sollte die Webseite ohne Logik in dem \gls{browser} laufen, könnte die Entscheidung auf eine traditionelle Webseite in Betracht gezogen werden.
% Muss die Webseite Inhalte ohne den \gls{server} darstellen, wäre eine \ac{SPA} der traditionellen Webseite vorzuziehen.