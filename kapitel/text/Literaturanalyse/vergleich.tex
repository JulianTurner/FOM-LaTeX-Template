\subsection{Stärken und Schwächen von Singe Page Applikationen und traditionellen Websites}
% Dieses Kapitel listet die Stärken und Schwächen von \ac{SPA} und traditionellen Websites auf.
% Die Stärken und Schwächen werden in den folgenden Kapiteln verwendet, um die Forschungsfrage zu beantworten.

\subsubsection*{Stärken von \ac{SPA}:}
\begin{itemize}
  \item Die gesamte Anwendung wird in den \gls{browser} geladen und muss nicht neu geladen werden, wenn ein Nutzer mit der Seite interagiert \cite{Smith2022}, \cite{Flanagan2011}
  \item Die Anwendung kann sich auf die Kommunikation mit Nutzdaten beschränken \cite{Smith2022}, \cite{Flanagan2011}
  \item Die Nutzerinteraktionen werden direkt im Browser verarbeitet und die Webseite muss nicht neu geladen werden \cite{Smith2022}, \cite{Flanagan2011}
  \item Eine \ac{SPA} kann einzelne Teile der Webseite unabhängig von anderen Teilen der Webseite dynamisch aktualisieren \cite{Smith2022}, \cite{Irudayaraj2019}
  \item Kann auch ohne Verbindung im Server funktionieren, wenn dafür ausgelegt \cite{Smith2022}, \cite{Gavrila2019}
  \item \gls{server} wird nicht zur Präsentation der Daten benötigt \cite{Flanagan2011}
\end{itemize}


\subsubsection*{Schwächen von \ac{SPA}:}

\begin{itemize}
  \item Die komplette Anwendung muss vom \gls{browser} geladen werden, bevor der Nutzer mit der Anwendung interagieren kann \cite{Smith2022}, \cite{Gavrila2019}
  \item Eine \ac{SPA} ist aufwendig zu entwickeln, da sie eine komplexere Softwarearchitektur benötigt \cite{Smith2022}, \cite{Flanagan2011}
  \item Eine \ac{SPA} bindet sich an ein bestimmtes \gls{framework}. \cite{Smith2022}
  \item Funktionieren nicht ohne Logik im \gls{browser} \cite{Smith2022}
  \item \ac{URL} Routing ist komplex und aufwendig umzusetzen \cite{Smith2022}
\end{itemize}

\subsubsection*{Stärken von traditionellen Websites:}

\begin{itemize}
  \item bessere \ac{SEO} da jede Seite einzeln geladen wird und der Inhalt der Seite direkt im \ac{HTML} steht \cite{Smith2022}, \cite{Irudayaraj2019}
  \item einfach gehaltene Webseiten, ohne starke Nutzerinteraktionen sind einfacher zu entwickeln \cite{Smith2022}, \cite{Irudayaraj2019}
  \item Funktioniert ohne Logik im \gls{browser} \cite{Smith2022}
\end{itemize}

\subsubsection*{Schwächen von traditionellen Websites:}

\begin{itemize}
  \item Jede Anfrage an den \gls{server} resultiert in einer neuen Webseite, welche vom \gls{server} an den \gls{browser} gesendet wird \cite{Flanagan2011}, \cite{Robbins2018} \cite{Solovei2018}
  \item neue Inhalte lassen sich nur mit einer kompletten Erneuerung des \ac{UI} darstellen \cite{Flanagan2011}, \cite{Robbins2018}
  \item \gls{server} wird zur Präsentation der Daten benötigt \cite{Flanagan2011}, \cite{Robbins2018}
  \item Kommunikation zwischen der traditionellen Webseite und dem \gls{server} hat einen \gls{overhead} \cite{Flanagan2011}
\end{itemize}

% \subsubsection*{Zusammenfassung der Stärken und Schwächen von \ac{SPA} und traditionellen Websites}
