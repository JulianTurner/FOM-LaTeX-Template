\section{Stärken und Schwächen von \ac{SPA} und traditionellen Websites}
Dieses Kapitel listet die Stärken und Schwächen von \ac{SPA} und traditionellen Websites auf.
Die Stärken und Schwächen werden in den folgenden Kapiteln verwendet, um die Forschungsfrage zu beantworten.

\subsection{Stärken von \ac{SPA}:}
\begin{itemize}
    \item Die gesamte Anwendung in den \gls{browser} geladen und wird nicht neu geladen, wenn ein Nutzer mit der Seite interagiert \footcite[Vgl. ][Seite 3]{Smith2022} \footcite[Vgl.][Seite 7]{Flanagan2011}
    \item Die Anwendung kann sich auf die Kommunikation mit Nutzdaten beschränken. \footcite[Vgl. ][Seite 3]{Smith2022} \footcite[Vgl.][Seite 13]{Flanagan2011}
    \item Die Nutzerinteraktionen werden direkt im Browser verarbeitet und die Webseite muss nicht neu geladen werden \footcite[Vgl. ][Seite 3]{Smith2022} \footcite[Vgl.][Seite 12]{Flanagan2011}
    \item Eine \ac{SPA} kann einzelne Teile der Webseite unabhängig von anderen Teilen der Webseite dynamisch aktualisieren \footcite[Vgl. ][Seite 7]{Smith2022} \footcite[Vgl.][Seite 9]{Flanagan2011}
    \item Kann auch ohne Verbindung im Server funktionieren, wenn dafür ausgelegt \footcite[Vgl. ][Seite 7]{Smith2022}
    \item \gls{server} wird nicht zur Präsentation der Daten benötigt \footcite[Vgl. ][Seite 7]{Flanagan2011}
\end{itemize}


\subsection{Schwächen von \ac{SPA}:}

\begin{itemize}
    \item Die komplette Anwendung muss vom \gls{browser} geladen werden, bevor der Nutzer mit der Anwendung interagieren kann \footcite[Vgl. ][Seite 3]{Smith2022}
    \item Eine \ac{SPA} ist aufwendig zu entwickeln, da sie eine komplexere Softwarearchitektur benötigt \footcite[Vgl. ][Seite 4]{Smith2022} \footcite[Vgl.][Seite 15]{Flanagan2011}
    \item Eine \ac{SPA} bindet sich an ein bestimmtes Framework. \footcite[Vgl. ][Seite 4]{Smith2022}
    \item Funktionieren nicht ohne Logik im \gls{browser} \footcite[Vgl. ][Seite 7]{Smith2022}
    \item \ac{URL} Routing ist komplex und aufwendig umzusetzen \footcite[Vgl. ][Seite 7]{Smith2022}
\end{itemize}

\subsection{Stärken von traditionellen Websites:}

\begin{itemize}
    \item bessere \ac{SEO} da jede Seite einzeln geladen wird und der Inhalt der Seite direkt im \ac{HTML} steht \footcite[Vgl. ][Seite 7]{Smith2022} \footcite[Vgl.][Seite 143]{Irudayaraj2019}
    \item einfach gehaltene Webseiten, ohne starke Nutzerinteraktionen sind einfacher zu entwickeln \footcite[Vgl. ][Seite 7]{Smith2022} \footcite[Vgl.][Seite 143]{Irudayaraj2019}
    \item Funktioniert ohne Logik im \gls{browser} \footcite[Vgl. ][Seite 7]{Smith2022}
\end{itemize}

\subsection{Schwächen von traditionellen Websites:}

\begin{itemize}
    \item Jede Anfrage an den \gls{server} resultiert in einer neuen Webseite, welche vom \gls{server} an den \gls{browser} gesendet wird. \footcite[Vgl. ][Seite 5]{Flanagan2011} \footcite[Vgl. ][Seite 33]{Robbins2018} \footcite[Vgl.][Seite 5]{Solovei2018}
    \item neue Inhalte lassen sich nur mit einer kompletten Erneuerung des \ac{UI} darstellen \footcite[Vgl. ][Seite 6]{Flanagan2011} \footcite[Vgl. ][Seite 33]{Robbins2018}
    \item \gls{server} wird zur Präsentation der Daten benötigt \footcite[Vgl. ][Seite 7]{Flanagan2011} \footcite[Vgl. ][Seite 33]{Robbins2018}
    \item Kommunikation zwischen der traditionellen Webseite und dem \gls{server} hat einen \gls{overhead} \footcite[Vgl. ][Seite 7]{Flanagan2011}
\end{itemize}

\subsection*{Zusammenfassung der Stärken und Schwächen von \ac{SPA} und traditionellen Websites}
Eine \ac{SPA} kann ihre Stärken bei der Entwicklung von komplexen Webanwendungen ausspielen.
Eine \ac{SPA} ist komplexer zu entwickeln, bietet aber Vorteile wie einen geringeren \gls{overhead} bei der Kommunikation mit dem \gls{server}.
Die Schwächen einer \ac{SPA} sind, dass die komplette Anwendung vom \gls{browser} geladen werden muss, bevor der Nutzer mit der Anwendung interagieren kann
und das Routing komplexer und aufwendiger als im Vergleich zu traditionellen Webseiten ist.
Eine traditionelle Webseite ist einfacher zu entwickeln und Funktioniert ohne Logik im \gls{browser}.
Durch das funktionieren ohne Logik im \gls{browser}, kann eine \gls{suchmaschine} eine traditionelle Webseite besser indexieren.
Die Schwächen einer traditionellen Webseite sind, dass bei jeder Anfrage an den \gls{server} eine neue Webseite geladen wird und der \gls{server} zur Präsentation der Daten benötigt wird.