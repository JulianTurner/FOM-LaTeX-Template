\section{Fazit}
Das abschließende Kapitel dieser Arbeit zieht Bilanz aus den gewonnenen Erkenntnissen und reflektiert die zentralen Ergebnisse der Untersuchung.
Es bietet eine kritische Betrachtung der vorgelegten Analysen und hebt die Bedeutung der Ergebnisse für die Praxis hervor.
Darüber hinaus werden mögliche Limitationen der Arbeit und Anregungen für zukünftige Forschungen in diesem Bereich vorgestellt.

\subsection{Reflexion der Arbeit}
Das Ziel der Arbeit war es, die Herausforderungen der Entwickler von \ac{SPA} Websites im Vergleich zu traditionellen Mehrseiten-Websites zu identifizieren.
Die Herausforderungen wurden anhand einer Literaturrecherche identifiziert.
Die Literaturrecherche wurde anhand eines iterativen Prozesses nach \Citeauthor{conf/ecis/BrockeSNRPC09} durchgeführt.
Durch diesen Prozess wurde sichergestellt, dass die Literaturrecherche nachvollziehbar und reproduzierbar ist.
Eine Herausforderung der Literaturrecherche war es, die relevanten Literaturquellen zu finden.
Anfangs wurde kaum relevante Literatur zu dem Thema gefunden oder nur sehr oberflächliche Informationen zu dem Thema.
Neben den oberflächlichen Informationen wurden auch viele Lehrbücher gefunden.
Zudem wurde festgestellt, dass nicht alle verfügbaren Literaturquellen berücksichtigt werden können.
Weiterhin wurde festgestellt, dass es keine einheitliche Definition für \ac{SPA} oder traditionelle Webseiten gibt.
Abschließend gibt es noch weitere Herausforderungen, in der Performance oder \ac{SEO} welche nicht behandelt wurden.

\subsection{Bedeutung der Ergebnisse}
Die Entscheidung zwischen der Implementierung einer \ac{SPA} und einer traditionellen Website hat direkte Auswirkungen auf die Praxis der Webentwicklung und das Nutzererlebnis.
Eine \ac{SPA} bietet eine reibungslose Nutzererfahrung, da sie das erneute Laden der gesamten Webseite vermeidet.
Dies kann zu einer schnelleren Interaktion und einer verbesserten Benutzerzufriedenheit führen, insbesondere bei Anwendungen, bei denen Benutzer häufig zwischen verschiedenen Abschnitten oder Funktionen wechseln.
Die Implementierung einer \ac{SPA} erfordert spezifische Kenntnisse in einem bestimmten \gls{framework}.
Dies kann zu einer steileren Lernkurve für Entwickler führen, die mit diesen Technologien nicht vertraut sind.
\ac{SPA} können den \gls{overhead} der Kommunikation mit dem \gls{server} reduzieren, da sie sich auf den Austausch von Nutzdaten beschränken können.
Dies kann zu einer effizienteren Nutzung und einer verbesserten Performance führen.
Traditionelle Websites können in der Regel leichter von einer \gls{suchmaschine} indexiert werden, da sie ohne Logik im \gls{browser} funktionieren.
Dies kann für Unternehmen von entscheidender Bedeutung sein, die auf \ac{SEO} angewiesen sind, um Sichtbarkeit zu erhöhen.
\subsection{Ausblick}
Eine künftige Forschung könnte sich mit einem hybriden Ansatz beschäftigen, in dem die Vorteile von \ac{SPA} und traditionellen Webseiten kombiniert werden.
Hier könnte die initiale Seite als traditionelle Webseite umgesetzt werden und die weiteren Seiten als \ac{SPA} betrieben.
Eine weitere Forschung könnte sich mit der Frage beschäftigen,
ob die Herausforderungen der Entwickler von \ac{SPA} Websites im Vergleich zu traditionellen Mehrseiten-Websites durch die Verwendung von einem \gls{framework} gelöst werden können.

\subsection{Zusammenfassung}
In der vorliegenden Arbeit wurde die Entwicklung und Implementierung von \ac{SPA} im Vergleich zu traditionellen Mehrseiten-Websites untersucht.
Während eine \ac{SPA} den Vorteil bietet, eine reibungslose Nutzererfahrung ermöglichen und den Kommunikations-Overhead mit dem \gls{server} minimieren, erfordert ihre Entwicklung eine tiefere Kenntnis spezifischer \gls{framework}s und Architekturen.
Traditionelle Websites hingegen sind einfacher zu entwickeln und werden von einer \gls{suchmaschine} besser indexiert, da sie ohne Logik im \gls{browser} funktionieren.
Allerdings resultiert jede Anfrage an den \gls{server} in einer neuen Webseite, was zu einem \gls{overhead} führt.
Die Wahl zwischen einer \ac{SPA} und einer traditionellen Website sollte daher auf der Art der Anwendung, den spezifischen Anforderungen und dem beabsichtigten Publikum basieren.
Zukünftige Forschungen könnten einen hybriden Ansatz in Betracht ziehen, der die Vorteile beider Technologien kombiniert.